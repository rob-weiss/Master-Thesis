% ******************************************************************
% Abstract
% ******************************************************************

\pdfbookmark[1]{Abstract}{Abstract}
\begingroup
\let\clearpage\relax
\let\cleardoublepage\relax
\let\cleardoublepage\relax

\chapter*{Abstract}

When creating mobile robots that operate autonomously, possibly without any human supervision, accurate self-localisation is one of the fundamental abilities such robots should possess. In conventional Monte Carlo localisation, a probability distribution over a space of possible hypotheses accommodates the inherent uncertainty in the position estimate, whereas bounded-error localisation provides a region that is guaranteed to contain the robot. However, this guarantee is accompanied by a constant probability over the confined region and therefore, depending on its size, the information yield may not be sufficient for certain practical applications. In this thesis, four new hybrid localisation algorithms are proposed, combining probabilistic filtering with non-linear bounded-error state estimation based on interval analysis. A forward-backward contractor and the Set Inverter via Interval Analysis are hybridised with a bootstrap filter and an unscented particle filter, respectively. The four new algorithms are applied to global localisation of an underwater robot, using simulated distance measurements to distinguishable landmarks. As opposed to previous hybrid methods found in the literature, the bounded-error state estimate is not maintained throughout the whole estimation process. Instead, it is only computed once in the beginning, when solving the wake-up robot problem, and after kidnapping of the robot, which drastically reduces the computational cost when compared to existing algorithms. Evaluating the performance of the localisation algorithms with different numbers of landmarks, it is shown that the newly proposed algorithms can solve the wake-up robot problem as well as the kidnapped robot problem more accurately than the two conventional probabilistic filters.

\vfill

\endgroup			

\vfill
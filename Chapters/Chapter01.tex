%************************************************
\chapter{Introduction}\label{ch:introduction}
%************************************************

For a long time, people have dreamt of building intelligent machines to perform tedious, repetitive or dangerous tasks. Today, we call these machines robots, derived from the Slavic word \emph{robota},  meaning servitude or drudgery \cite{choset2005principles}. For a mobile robot, in order to operate in its environment, it is essential to know its position relative to an external reference frame. The corresponding localisation problem constitutes the most basic perceptual challenge in robotics and is described in more detail below.

\section{The Global Localisation Problem}

 A scenario in which a robot is given a map of its environment, to estimate its position relative to this map using its sensors, is commonly known as the \emph{global localisation problem}. In contrast, \emph{position tracking} denotes the process of continuously determining the robot's position relative to a known initial position. Since here the uncertainty is confined to the region near the robot's true position, tracking is a \emph{local} problem and easier than global localisation. When a robot has to establish its own global position without prior knowledge, for instance after having been carried to an arbitrary location before being put to operation, this is referred to as the \emph{wake-up robot problem}. The related \emph{kidnapped robot problem}, which forms another subclass of global localisation problems, describes a situation where a well-localised mobile robot is teleported to an arbitrary location without being told. That is, the robot strongly believes itself to be somewhere else at the time of the kidnapping. Solving the above problems under varying circumstances using different sensory input represents a major challenge in robotics and is of paramount importance for a successful practical application of mobile robots.
 
 \subsection{Sensor Systems}
 
 In tackling the self-localisation problems described above, the fundamental choices of sensory information may be classified as those obtained by \emph{proprioceptive} sensors, such as wheel encoders or inertial sensors, and \emph{exteroceptive} sensors, such as cameras, laser scanners or supersonic sensors. By means of the latter type of sensors, the robot can sense \emph{landmarks}, which denote any identifiable sensory perception that has a known position with respect to a given global coordinate system. In practice, proprioceptive methods alone fail after a short time since they are impaired by incorrigible drift errors. For this reason, a fusion of proprioceptive and exteroceptive information has been widely and successfully used in mobile robotics \cite{nehmzow2001mobile}.

\subsection{Sensor Fusion}

As noise-free sensors for measuring the position do not exist, the position has to be inferred from the evolution of noisy sensor data over time. The combination of information from multiple sensors with the aim to increase the overall precision of the estimation of a certain quantity of interest is termed \emph{sensor fusion}. \citeauthor{raol2009multi} \cite{raol2009multi} states the following advantages of sensor fusion:
 
\begin{itemize}
\item Robust functional and operational performance is given, in case of data loss from one sensor, due to redundancy provided by multiple sensors.
\item Enhanced confidence in the results inferred from the measurement of one sensor, if they are confirmed by the measurement of another sensor.
\item With sensor fusion an arbitrary fine time resolution of measurements is possible, whereas single sensors need a finite time to transmit measurements and so limit the frequency of measurements.
\item One sensor might be, to some extent, better in a certain state of the measured process and thus, by fusing multiple sensor signals, a satisfactory accuracy among all states of a process may be attained.
\end{itemize}

\noindent
As we shall see in Chapter \ref{ch:theoretical_background}, in the context of self-localisation, digital filters can be used to fuse information from multiple sensors with additional a priori information about the position of a mobile robot.


\section{Motivation}

Robotics has enormous potential to change the world for the better, beckoning humanity to tap into it. Releasing mankind from mundane and repetitive tasks frees time and energy and thus allows us to focus on the more interpersonal and creative aspects of life. Enhanced efficiency enables us to engage with higher order functions and to devote our time to further improving quality of life. Automating dangerous tasks protects human lives and renders operation under inhuman conditions possible in the first place. Robotics' potential to make a positive, lasting impact on the world was the motivation for this work. 

\section{State of the Art}

Mobile robots have successfully been applied to search and rescue \cite{s17102426, 7485707, Beck:2016:OPC:2936924.2937074, voyles2008search, shah2004survey}, monitoring and surveillance \cite{dunbabin2012robots}, autonomous locomotion \cite{bresson2017simultaneous, vivacqua2017low}, and education \cite{crnokic_applications}. When aiming at creating mobile robots that operate autonomously and safely in these scenarios, possibly without any human supervision, accurate self-localisation is one of the fundamental abilities any mobile robot should possess. In order to perform robust localisation, many novel methods have emerged in the past few years, based both on probability theory and interval analysis. Nevertheless, self-localisation remains a rich and highly active area of research to this day. 

\subsection{Probabilistic Localisation}

Taking a probabilistic approach to localisation, uncertainty is represented explicitly, using the calculus of probability theory. In \emph{probabilistic localisation} a probability distribution over a space of possible hypotheses accommodates the inherent uncertainty in a position estimate. This uncertainty originates from the above-mentioned measurement noise and other influences discussed in more detail in the following chapter. In the literature, there are numerous probabilistic methods, applied to both global localisation as well as tracking. Besides a manifold of particle filters \cite{ko2012particle, THRUN200199, roumeliotis2000, yong_better_proposal2001}, Kalman filters \cite{jensfelt_globalloc2001, roumeliotis2000} have been used successfully.  


\subsection{Bounded-Error Localisation}

Naturally, the probabilistic approach does not ensure a correct solution and in fact probabilistic estimators may diverge under certain circumstances. Hence, several methods based on interval computations have been developed to tackle the local and global localisation problem \cite{le2012set, meizel2002initial, seignez_interval2008, Kieffer2000, kieffer1999guaranteed, seignez2009real, jaulin2009, jaulin2006localization, 5354673}. As opposed to the probabilistic approach, which provides a point estimate of the position, the rationale of the so-called \emph{bounded-error localisation} is to determine a region that is guaranteed to contain the robot's position. Here, however, the guarantee is accompanied by a constant probability over the confined region and in practice the information yield may not be sufficient.

\subsection{Hybrid Localisation}

In addition to stand-alone probabilistic or bounded-error localisation methods, there have been attempts to combine both in order to mitigate their respective shortcomings and thus improve the localisation accuracy. \citeauthor{neulandh_ybridazation} \cite{neulandh_ybridazation} used a particle filter in combination with a set-membership method to restrict the spread of particles to regions of the search space that are associated with a high observation likelihood. In \cite{neuland_set_inversion} they used a set-inversion method instead and reported an increase in the estimation accuracy in return for higher computational cost in both cases. 

\citeauthor{nicol2017} \cite{nicol2017} combined a set-membership method with a Kalman filter to obtain a reliable and precise algorithm for simultaneous localisation and mapping of underwater robots. \citeauthor{ashokaraj2004sensor} proposed sensor-based robot localisation using an extended Kalman filter \cite{ashokaraj2004sensor} as well as an unscented Kalman filter \cite{ashokaraj2004sensorukf} in combination with interval analysis to bound the estimation error in the presence of landmarks. If the position estimate of the Kalman filter lay outside of the feasible region it was corrected to the geometrically closest point on the boundary. In \cite{ashokaraj2004fuzzy}, multiple interval robot positions were processed using a fuzzy logic weighted average algorithm to obtain a single robot interval position. The error of an unscented Kalman filter position estimate was then bounded by the interval robot position as described above. In \cite{ashokaraj2004mobile}, \citeauthor{ashokaraj2004mobile} used ultrasonic sensors with limited range and corrected the mean and covariance of an unscented Kalman filter by means of the interval estimate. All their methods resulted in a more accurate position estimate.

\section{Contributions} 

As will be shown in the following chapter, in Monte Carlo localisation there is a natural dilemma between accuracy and computational cost, which can be balanced by the number of hypotheses used, also referred to as particles. Both bounded-error state estimation as well as an unscented Kalman filter can be utilised in order to move particles to regions of the state space that are associated with a high observation likelihood. As no particles are wasted in unlikely regions, with a finite number of particles the particle density in likely regions can be increased and consequently the localisation accuracy can be improved. Previous hybrid localisation methods successfully combined bounded-error state estimation with a bootstrap particle filter. However, it is shown experimentally below that when too little information in terms of visible landmarks are available, the hybridisation of both bounded-error state estimation and an unscented particle filter can further improve the estimation accuracy. %Furthermore, available additional information in the form of constraints may improve the localisation accuracy.

 A particle filter and an unscented particle filter is combined with two bounded-error estimators, namely a forward-backward contractor and the Set Inverter via Interval Analysis, respectively. The four resulting new hybrid localisation algorithms and the two conventional probabilistic filters are applied in three different simulated landmark-based global localisation scenarios. While the hybrid localisation methods in \cite{neulandh_ybridazation} and \cite{neuland_set_inversion} maintain both a bootstrap particle filter estimate and a bounded-error state estimate at each time step, in order to bound the estimation error of the particle filter, it is shown that the localisation accuracy particularly benefits from the bounded-error state estimate in the very first iterations or shortly after kidnapping of the robot, respectively. The bounded-error state estimate is therefore not maintained throughout the whole estimation process. Instead, the satisfaction of given constraints based on geometrical considerations of the environment are tested, in order to bound the estimation error and detect kidnapping. In the latter case, the bounded-error state estimation is triggered again to repeat the global localisation process over the entire map. The rationale behind the newly proposed algorithms, which are explained in detail in Chapter \ref{ch:implementation}, is to drastically reduce computational cost when compared to previous methods while preserving the benefits of the hybrid approach and therefore improve the estimation accuracy when compared to conventional unconstrained probabilistic filtering.
 

\section{Methodology}

In the following chapter, stochastic filtering theory is reviewed in detail, beginning with a statement of the problem and its conceptual solution, the Bayesian filter. Under linear-quadratic-Gaussian circumstances, the Kalman filter represents an optimal solution within the Bayesian framework. As the linear scenario is hardly found in practice, suboptimal non-linear filtering techniques such as the extended Kalman filter and the unscented Kalman filter are extensively investigated. Subsequently, we focus our attention on the sequential Monte Carlo method, including importance sampling and resampling. As another representative of the Bayesian framework, the generic particle filter assumes a non-linear, non-Gaussian model. Using the transition prior probability distribution as a proposal distribution yields the bootstrap filter, while the unscented particle filter represents a more elaborate estimator that uses an unscented Kalman filter in order to generate better proposal distributions. At the end of Chapter \ref{ch:theoretical_background}, we introduce basic concepts of interval analysis and two algorithms that can be applied to bounded-error localisation. Chapter \ref{ch:implementation} elaborates on the four newly proposed localisation algorithms. The system and measurement models of both the probabilistic and bounded-error estimators are described and the incorporation of constraints into the inherently unconstraint bayesian filters is explained. In Chapter \ref{ch:experiments} the four new hybrid localisation algorithms and the two conventional probabilistic filters are put to the test in three different simulated landmark-based global localisation scenarios. A detailed explanation of the experiments is followed by the presentation of the results. Finally, in Chapter \ref{ch:Conclusion and Future Work} we conclude and present potential future work.





% ******************************************************************
% Configuration
% ******************************************************************

\PassOptionsToPackage{eulerchapternumbers, listings, pdfspacing, subfig, beramono, eulermath}{classicthesis}                                        


% ******************************************************************
% Personal Data
% ******************************************************************

\newcommand{\myTitle}{Hybridisation of Sequential Monte Carlo Simulation with Non-linear Bounded-error State Estimation Based on Interval Analysis Applied to Global Localisation of Mobile Robots
\xspace} 
%Hybridisation of an Unscented Particle Filter with Bounded-error Estimation Based on Interval Analysis Applied to Global Localisation of Mobile Robots
%Comparison of Probabilistic, Bounded-error and Hybrid Techniques for Global Localisation of Mobile Robots % set-membership methods
\newcommand{\mySubtitle}{A thesis submitted in partial fulfilment of the requirements for \\ the degree of Master of Science in Computer Science\xspace}
\newcommand{\myDegree}{Master of Science\xspace}
\newcommand{\myName}{Robin Weiss\xspace}
\newcommand{\myProf}{Prof. Dr.-Ing. Peter Glösekötter\xspace}
\newcommand{\mySupervisor}{Prof. Dr. Mariana Luderitz Kolberg\xspace} % Edson Prestes
\newcommand{\myDepartment}{Department of Electrical Engineering \\
\vspace{-0.3cm} and Computer Science\xspace}
\newcommand{\myUni}{Münster University of Applied Sciences\xspace}
\newcommand{\myLocation}{$\Phi$-Robotics Research Lab, Federal University of Rio Grande do Sul, Porto Alegre, Brazil\xspace}
\newcommand{\myTime}{November 2018\xspace}%\monthname\ \the\year
\newcommand{\myVersion}{Version 1.0\xspace}
\newcommand{\myURL}{http://www.robinweiss.de/static/pdf/master-thesis.pdf}


% ******************************************************************
% Fine Tuning
% ******************************************************************

\newcounter{dummy} % For correct hyperlinks to index, bib, etc.
\newlength{\abcd} % for ab..z string length calculation
\providecommand{\mLyX}{L\kern-.1667em\lower.25em\hbox{Y}\kern-.125emX\@}
\newcommand{\ie}{i.\,e.}
\newcommand{\eg}{e.\,g.}

% ******************************************************************
% Load Packages
% ******************************************************************

\usepackage[british, ngerman]{babel}                  
\usepackage{csquotes}
\usepackage[backend=bibtex8, bibencoding=ascii, language=auto, style=numeric-comp, sorting=none, maxbibnames=10, natbib=true]{biblatex}

\usepackage[T1]{fontenc}
%\renewcommand*\familydefault{\sfdefault} %% Only if the base font of the document is to be sans serif
\usepackage{textcomp}
\usepackage{datetime}

% Some fixes:
\usepackage{scrhack}
\usepackage{xspace}
\usepackage{mparhack}
\usepackage{fixltx2e}

% Tables, Figures, Captions:
\usepackage{tabularx}
\setlength{\extrarowheight}{3pt} % increase table row height

\usepackage{makecell}
\renewcommand\theadalign{bl}
\renewcommand\theadgape{\Gape[4pt]}
\renewcommand\cellgape{\Gape[4pt]}

\usepackage{booktabs}
\newcommand{\ra}[1]{\renewcommand{\arraystretch}{#1}}

\newcommand{\tableheadline}[1]{\multicolumn{1}{c}{\spacedlowsmallcaps{#1}}}
\newcommand{\myfloatalign}{\centering} % to be used with each float for alignment

\usepackage{caption}
\captionsetup{font=small}
\usepackage{subfigure}

\usepackage{listings}
\lstset{language=[LaTeX]Tex,
    morekeywords={PassOptionsToPackage, selectlanguage},
    keywordstyle=\color{RoyalBlue},
    basicstyle=\small\ttfamily,
    commentstyle=\color{Green}\ttfamily,
    stringstyle=\rmfamily,
    numbers=left,
    numberstyle=\scriptsize\color{gray},%\tiny
    stepnumber=2,
    numbersep=8pt,
    showstringspaces=false,
    breaklines=true,
    %frameround=tttt,
    %frame=single,
    belowcaptionskip=.75\baselineskip
} 

% Code hightlighting
\newcommand{\code}[1]{
	\tikzexternaldisable
	\tikz[baseline=(text.base)]{\node [rounded corners,fill=gray!12](text){\texttt{#1}};}
	\tikzexternalenable
}%

% PDFLaTeX, Hyperreferences and Citation
\usepackage[pdftex, hyperfootnotes=false, pdfpagelabels]{hyperref}
\pdfcompresslevel=9
\pdfadjustspacing=1 

\usepackage[pdftex]{graphicx} 
\usepackage{lscape}
 
\hypersetup{
    colorlinks=true,
    linktocpage=true,
    pdfstartpage=3,
    pdfstartview=FitV,%
    % Uncomment for printing
    %colorlinks=false, linktocpage=false, pdfstartpage=3, pdfstartview=FitV, pdfborder={0 0 0},%
    breaklinks=true,
    pdfpagemode=UseNone,
    pageanchor=true,
    plainpages=false,
    bookmarksopen=true,
    bookmarksopenlevel=1,
    hypertexnames=true,
    pdfhighlight=/O,
    urlcolor=black,
    linkcolor=black,
    citecolor=black,
    pdftitle={\myTitle},
    pdfauthor={\textcopyright\ \myName, \myUni},
    pdfsubject={},
    pdfkeywords={},
    pdfcreator={pdfLaTeX},
    pdfproducer={LaTeX with hyperref and classicthesis}%
}   


\makeatletter
\@ifpackageloaded{babel}
    {
       \addto\extrasbritish{
			\renewcommand*{\figureautorefname}{Figure}
			\renewcommand*{\tableautorefname}{Table}
			\renewcommand*{\partautorefname}{Part}
			\renewcommand*{\chapterautorefname}{Chapter}
			\renewcommand*{\sectionautorefname}{Section}
			\renewcommand*{\subsectionautorefname}{Section}
			\renewcommand*{\subsubsectionautorefname}{Section}   
                }
       \addto\extrasngerman{
			\renewcommand*{\paragraphautorefname}{Absatz}
			\renewcommand*{\subparagraphautorefname}{Unterabsatz}
			\renewcommand*{\footnoteautorefname}{Fu\"snote}
			\renewcommand*{\FancyVerbLineautorefname}{Zeile}
			\renewcommand*{\theoremautorefname}{Theorem}
			\renewcommand*{\appendixautorefname}{Anhang}
			\renewcommand*{\equationautorefname}{Gleichung}      
			\renewcommand*{\itemautorefname}{Punkt}
                }
            \providecommand{\subfigureautorefname}{\figureautorefname}%             
    }{\relax}
\makeatother

\listfiles

\usepackage{classicthesis}

% ******************************************************************
% Fonts
% ******************************************************************

%\usepackage[oldstylenums]{kpfonts} % oldstyle notextcomp
%\usepackage[osf]{libertine}
%\usepackage[light,condensed,math]{iwona}
%\renewcommand{\sfdefault}{iwona}
%\usepackage{lmodern} % <-- no osf support :-(
%\usepackage{cfr-lm} % 
%\usepackage[urw-garamond]{mathdesign} % <-- no osf support :-(

% ******************************************************************
% Mathematics
% ******************************************************************

\usepackage{bm}
\usepackage{amsmath}
\usepackage{amsfonts}
\usepackage{amssymb}
\usepackage{mathtools}
\usepackage{units}
\usepackage{algorithm}
\usepackage{algpseudocode}

\algnewcommand{\algorithmicgoto}{\textbf{go to}}%
\algnewcommand{\Goto}[1]{\algorithmicgoto~\ref{#1}}%

%\renewcommand{\algorithmname}{Algorithm}
%\renewcommand{\ALG@name}{Algorithm}
%\renewcommand{\algorithmcfname}{Algorithm}
%\floatname{algorithm}{MegaAlgorithm}
%\makeatletter
%\let\algorithmname\fname@algorithm
%\makeatother
\renewcommand{\algorithmicrequire}{\textbf{Input:}}
\renewcommand{\algorithmicensure}{\textbf{Output:}}

% ******************************************************************
% TikZ and PGFplot
% ******************************************************************

\usepackage{tikz}
\usepackage{tikzscale}
\usepackage{tikz-3dplot}
\usetikzlibrary{external, trees, shapes, arrows, calc, patterns, decorations.pathmorphing, decorations.markings, positioning}\tikzexternalize[prefix=Tikz/]
\tikzsetexternalprefix{Figures/Tikz/}
%\tikzset{external/force remake} % Remake all tikz pictures.
\maxdeadcycles3000
\usepackage{morefloats}

\newlength\figureheight 
\newlength\figurewidth 

\usepackage{pgfplots}
\pgfplotsset{
    tick label style={font=\scriptsize},
    label style={font=\small},
	title style={font=\normalsize\normalfont},
    legend style={font=\scriptsize}
}

% ******************************************************************
% Acronyms and Notation
% ******************************************************************

\usepackage[acronym, nopostdot, nonumberlist]{glossaries}
\newglossary[nlg]{notation}{not}{ntn}{Notation}


\newacronym[description={Unscented Kalman filter}]{UFK}{UKF}{unscented Kalman Filter}

\newacronym[description={Extended Kalman filter}]{EKF}{EKF}{extended Kalman filter}

\newacronym[description={Kalman filter}]{KF}{KF}{Kalman filter}

\newacronym[description={Sequential Monte Carlo}]{SMC}{SMC}{sequential Monte Carlo}

\newacronym[description={Sequential importance sampling}]{SIS}{SIS}{sequential importance sampling}

\newacronym[description={Sequential importance resampling}]{SIR}{SIR}{sequential importance resampling}

\newacronym[description={Probability density function}]{PDF}{PDF}{probability density function}


\newacronym[description={Cumulative distribution function}]{CDF}{CDF}{cumulative distribution function}

\newacronym[description={Constraint satisfaction problem}]{CSP}{CSP}{constraint satisfaction problem}

\newacronym[description={Numerical constraint satisfaction problem}]{NCSP}{NCSP}{numerical constraint satisfaction problem}

\newacronym[description={Particle filter}]{PF}{PF}{particle filter}

\newacronym[description={Unscented particle filter}]{UPF}{UPF}{unscented particle filter}

\newacronym[description={Autonomous underwater vehicle}]{AUV}{AUV}{autonomous underwater vehicle}



\newglossaryentry{not:B_v}{
  type=notation,
  name={$\bm{B}$},
  description={Control matrix that relates the control input to the state of a linear dynamical system},
  sort={Ba}}
 
 \newglossaryentry{not:n}{
  type=notation,
  name={$n$},
  description={Non-dimensional number},
  sort={n}}
  
\newglossaryentry{not:x_v}{
  type=notation,
  name={$\bm{x}$},
  description={State vector of a dynamical system},
  sort={xb}}
    
\newglossaryentry{not:w_v}{
  type=notation,
  name={$\bm{w}$},
  description={Process noise vector},
  sort={wa}}
  
\newglossaryentry{not:u}{
  type=notation,
  name={$\bm{u}$},
  description={Control input vector},
  sort={u}}
  
\newglossaryentry{not:v}{
  type=notation,
  name={$\bm{v}$},
  description={Measurement noise vector},
  sort={v}}
  
 \newglossaryentry{not:x_hat_v}{
  type=notation,
  name={$\hat{\bm{x}}$},
  description={Estimate of the state vector of a dynamical system},
  sort={xd}}
  
  \newglossaryentry{not:x_bar_v}{
  type=notation,
  name={$\bar{\bm{x}}$},
  description={Mean of the random vector $\bm{x}$},
  sort={xdd}}
  
\newglossaryentry{not:prediction_measurement}{
  type=notation,
  name={$\hat{\bm{z}}_{k|k-1}$},
  description={Prediction of the measurement vector $\bm{z}_k$ before it comes available at time $k$},
  sort={zaaaaa}}
  
\newglossaryentry{not:x_hat_minus}{
  type=notation,
  name={$\hat{\bm{x}}_{k|k-1}$},
  description={A priori estimate of $\hat{\bm{x}}_k$, conditioned on all prior measurements except the one at time $k$},
  sort={xe}}
  
\newglossaryentry{not:z}{
  type=notation,
  name={$\bm{z}$},
  description={Observation or measurement vector of a dynamical system},
  sort={zaa}}
  
  \newglossaryentry{not:z0}{
  type=notation,
  name={$\bm{z}_0$},
  description={Empty measurement},
  sort={zaaa}}
  
\newglossaryentry{not:H}{
  type=notation,
  name={$\bm{H}$},
  description={Measurement sensitivity matrix defining the linear relationship between the state of a dynamical system and its observation},
  sort={Haa}}
  
\newglossaryentry{not:H1}{
  type=notation,
  name={$\bm{H}^{[1]}$},
  description={Jacobian matrix of the function $\bm{h}$},
  sort={Haaa}}
  
\newglossaryentry{not:K_v}{
  type=notation,
  name={$\bm{K}$},
  description={Kalman gain matrix},
  sort={kaaa}}
  
 \newglossaryentry{not:k}{
  type=notation,
  name={$k$},
  description={Discrete time, $k \in \mathbb{N}^0$},
  sort={k}}
  
\newglossaryentry{not:P}{
  type=notation,
  name={$\bm{P}_{k}$},
  description={Error covariance matrix representing the uncertainty in the state estimation},
  sort={paa}}
  
  \newglossaryentry{not:Pinnov}{
  type=notation,
  name={$\bm{P}_{\tilde{\bm{z}}_k \tilde{\bm{z}}_k}$},
  description={Innovation covariance matrix},
  sort={paaaaa}}
  
  \newglossaryentry{not:Pcrosscov}{
  type=notation,
  name={$\bm{P}_{\tilde{\bm{x}}_k \tilde{\bm{z}}_k}$},
  description={Cross covariance matrix},
  sort={paaaaaa}}
  
 \newglossaryentry{not:Papri}{
  type=notation,
  name={$\bm{P}_{k|k-1}$},
  description={A priori error covariance matrix representing the uncertainty of the a priori estimate $\hat{\bm{x}}_{k|k-1}$ },
  sort={paaa}}
  
\newglossaryentry{not:Q}{
  type=notation,
  name={$\bm{Q}$},
  description={Covariance matrix of the process noise},
  sort={Q}}
  
\newglossaryentry{not:R}{
  type=notation,
  name={$\bm{R}$},
  description={Covariance matrix of the measurement noise},
  sort={R}}
  
\newglossaryentry{not:phi1}{
  type=notation,
  name={$\bm{\Phi}$},
  description={State transition matrix of a linear dynamical system},
  sort={phiaa}}
  
\newglossaryentry{not:phi}{
  type=notation,
  name={$\bm{\Phi}^{[1]}$},
  description={Jacobian matrix of the function $\bm{\phi}$},
  sort={phiaaa}}
  
\newglossaryentry{not:x_k}{
  type=notation,
  name={$\bm{x}_k$},
  description={The $k$th element of a sequence $\dots$, $\bm{x}_{k-1}$, $\bm{x}_k$, $\bm{x}_{k+1}, \dots$ of vectors},
  sort={xc}}
  
 \newglossaryentry{not:unit-delay}{
  type=notation,
  name={$z^{-1}$},
  description={Unit-delay},
  sort={za}}

\newglossaryentry{not:phi_vec}{
  type=notation,
  name={$\bm{\phi}$},
  description={Non-linear state transition function of a dynamical system},
  sort={phia}}
  
\newglossaryentry{not:h_vec}{
  type=notation,
  name={$\bm{h}$},
  description={Non-linear function that relates the measurement to the state of a dynamical system},
  sort={ha}}
  
\newglossaryentry{not:identity_matrix}{
  type=notation,
  name={$\bm{I}_n$},
  description={Identity matrix, $\bm{I}_n \in \mathbb{R}^{n \times n}$},
  sort={I}}  
  
\newglossaryentry{not:sampling interval}{
  type=notation,
  name={$\bm{w} \sim \mathcal{N}(\bm{\mu}, \bm{P})$},
  description={The random variable $\bm{w}$ is distributed normally with mean $\bm{\mu}$ and covariance $\bm{P}$},
  sort={waaa}}    
  

\newglossaryentry{not:sequence_x}{
  type=notation,
  name={$\bm{X}_k$},
  description={Sequence of states, denoting $\{\bm{x}_i\}^k_{i = 0}$},
  sort={xk}}   
  
  \newglossaryentry{not:sequence_u}{
  type=notation,
  name={$\bm{U}_k$},
  description={Sequence of control inputs, denoting $\{\bm{u}_i\}^k_{i = 0}$},
  sort={uk}}  
  
  \newglossaryentry{not:sequence_y}{
  type=notation,
  name={$\bm{Z}_k$},
  description={Sequence of observations, denoting $\{\bm{z}_i\}^k_{i = 1}$},
  sort={zk}}   
  
\newglossaryentry{not:initial_dist}{
  type=notation,
  name={$p(\bm{x}_0)$},
  description={Probability distribution of the initial state},
  sort={p0}}    
  
  \newglossaryentry{not:pred_dist}{
  type=notation,
  name={$p(\bm{x}_k\,|\,\bm{Z}_{k-1}, \bm{U}_{k-1})$},
  description={Predictive distribution of the state $\bm{x}_k$ at the current time $k$, given the entire sequence of observations and the entire sequence of control inputs up to and including time $k-1$},
  sort={p1}}    
  
\newglossaryentry{not:post_dist}{
  type=notation,
  name={$p(\bm{x}_k\,|\,\bm{Z}_{k}, \bm{U}_{k-1})$},
  description={Posterior distribution of the current state $\bm{x}_k$, given the entire sequence of observations up to and including the current time $k$ and the entire sequence of control inputs up to and including time $k-1$},
  sort={p2}}   
  
  \newglossaryentry{not:post_dist_normally}{
  type=notation,
  name={$p_{\mathcal{N}}(\bm{x}_k\,|\,\bm{Z}_{k}, \bm{U}_{k-1})$},
  description={Gaussian approximation of the posterior distribution of the current state $\bm{x}_k$, given the entire sequence of observations up to and including the current time $k$ and the entire sequence of control inputs up to and including time $k-1$},
  sort={p22}}  
  
  \newglossaryentry{not:post_dist_constraint}{
  type=notation,
  name={$p_C(\bm{x}_k\,|\,\bm{Z}_{k}, \bm{U}_{k-1})$},
  description={Constrained posterior distribution of the current state $\bm{x}_k$, given the entire sequence of observations up to and including the current time $k$ and the entire sequence of control inputs up to and including time $k-1$},
  sort={p21}}  
  
  
  \newglossaryentry{not:trans_dist}{
  type=notation,
  name={$p(\bm{x}_k\,|\,\bm{x}_{k-1}, \bm{u}_{k-1})$},
  description={State-transition distribution of the current state $\bm{x}_k$, given the immediate past state $\bm{x}_{k-1}$ and control input $\bm{u}_{k-1}$},
  sort={p3}} 
  
  \newglossaryentry{not:likelyhood_dist}{
  type=notation,
  name={$p(\bm{z}_k\,|\,\bm{x}_{k})$},
  description={Likelihood function of the current observation $\bm{z}_k$, given the current state $\bm{x}_{k}$},
  sort={p4}}  
  
  
 \newglossaryentry{not:expected}{
  type=notation,
  name={$\mathbb{E}[\bm{x}]$},
  description={Expected value of $\bm{x}$},
  sort={e}}  
  
  \newglossaryentry{not:partial}{
  type=notation,
  name={$\displaystyle \left. \frac{\partial \bm{f}(\bm{x})}{\partial \bm{x}} \right|_{\bm{x}=\bm{x}_{0}}$},
  description={Partial derivative of $\bm{f}$ with respect to $\bm{x}$, evaluated at $\bm{x}_0$},
  sort={Dervi}}  
  
    \newglossaryentry{not:sigmapoints}{
  type=notation,
  name={$\bm{\mathcal{X}}$},
  description={Set of sigma points, denoting $\big\{\mathcal{X}_i\big\}^{2n}_{i = 0}$},
  sort={Xzzzzzzz}}  
  
      \newglossaryentry{not:ndimreals}{
  type=notation,
  name={$\mathbb{R}^{n}$},
  description={$n$-dimensional real vector space},
  sort={Rr}}  
  
  
  \newglossaryentry{not:sigmapoint}{
  type=notation,
  name={$\mathcal{X}_i$},
  description={The $i$-th sigma point},
  sort={Xxxx}}  
  
   \newglossaryentry{not:matrixroot}{
  type=notation,
  name={$\Big(\sqrt{(n + \lambda) \bm{P}_{\bm{x}}}\Big)_i$},
  description={The $i$-th column of the matrix square root of $(n + \lambda) \bm{P}_{\bm{x}}$},
  sort={squareRoot}}  
  
     \newglossaryentry{not:alpha}{
  type=notation,
  name={$\alpha$},
  description={Scaling parameter that controls the spread of sigma points around the mean},
  sort={alpha}}  
  
       \newglossaryentry{not:kappa}{
  type=notation,
  name={$\kappa$},
  description={Scaling parameter in unscented transformation; $\kappa \geq 1$ guarantees positive semi-definiteness of the covariance matrix},
  sort={kappa}}  
  
  
 \newglossaryentry{not:weightmean}{
  type=notation,
  name={$W_i^{(m)}$},
  description={Weight of the $i$-th sigma point for the computation of the mean},
  sort={weightmean}} 
  
 \newglossaryentry{not:weightcov}{
  type=notation,
  name={$W_i^{(c)}$},
  description={Weight of the $i$-th sigma point for the computation of the covariance},
  sort={weightcov}} 
  

 \newglossaryentry{not:sigmapointold}{
  type=notation,
  name={$\bm{\mathcal{X}}_{k-1}$},
  description={Set of sigma points computed using the previous state estimate},
  sort={Xxxxx}} 
  
   \newglossaryentry{not:sigmapointolda}{
  type=notation,
  name={$\bm{\mathcal{X}}^a_{k-1}$},
  description={Set of augmented sigma points computed using the previous state estimate},
  sort={Xxxxx}}
  
   \newglossaryentry{not:sigmapointoldx}{
  type=notation,
  name={$\bm{\mathcal{X}}^{\bm{x}}_{k|k-1}$},
  description={Set of sigma points constituted of the components that are associated with the state},
  sort={Xxxxxd}} 
  
   \newglossaryentry{not:sigmapointoldw}{
  type=notation,
  name={$\bm{\mathcal{X}}^{\bm{w}}_{k|k-1}$},
  description={Set of sigma points constituted of the components that are associated with the process noise},
  sort={Xxxxxc}} 
  
   \newglossaryentry{not:sigmapointoldv}{
  type=notation,
  name={$\bm{\mathcal{X}}^{\bm{v}}_{k|k-1}$},
  description={Set of sigma points constituted of the components that are associated with the measurement noise},
  sort={Xxxxxb}} 
  
  
   \newglossaryentry{not:sigmapointmeasure}{
  type=notation,
  name={$\bm{\mathcal{Z}}_{k|k-1}$},
  description={Set of sigma points capturing the predicted measurement},
  sort={Zzzzzzz}} 
 
  
   \newglossaryentry{not:expectation}{
  type=notation,
  name={$\mathbb{E}_{p(\bm{x}_k\,|\,\bm{Z}_{k}, \bm{U}_{k-1})}\big[f(\bm{x}_k)\big]$},
  description={Expectation under $p(\bm{x}_k\,|\,\bm{Z}_{k}, \bm{U}_{k-1})$},
  sort={eeee}} 
  
  
     \newglossaryentry{not:expectationest}{
  type=notation,
  name={$\hat{\mathbb{E}}\big[f(\bm{x}_k)\big]$},
  description={Estimate of the expectation $\mathbb{E}\big[f(\bm{x}_k)\big]$},
  sort={eeeee}} 
  
  
       \newglossaryentry{not:realinterval}{
  type=notation,
  name={$[x]$},
  description={Real interval, denoting $\big\{x \in \mathbb{R} \,\,|\,\, \underline{x} \leq x \leq \overline{x}\big\}$},
  sort={x}} 
  
  
       \newglossaryentry{not:realintervalvec}{
  type=notation,
  name={$[\bm{x}]$},
  description={Real interval vector, denoting the Cartesian product of intervals, $[x]_1 \times [x]_2 \times \dots \times [x]_n$},
  sort={xa}} 
  
  \newglossaryentry{not:realintervalwid}{
  type=notation,
  name={$\mathrm{w}\big([x]\big)$},
  description={Width of an interval $[x]$},
  sort={waa}} 
  
    \newglossaryentry{not:realintervalmid}{
  type=notation,
  name={$\mathrm{mid}\big([x]\big)$},
  description={Midpoint of an interval $[x]$},
  sort={mid}} 
  
   \newglossaryentry{not:realintervalabs}{
  type=notation,
  name={$\big|[x]\big|$},
  description={Absolute value of an interval $[x]$},
  sort={abs}} 
  
  
    \newglossaryentry{not:inclfunc}{
  type=notation,
  name={$[\bm{f}]\big([\bm{x}]\big)$},
  description={Inclusion function of $\bm{f}$},
  sort={faa}} 
  
  
      \newglossaryentry{not:sulutionset}{
  type=notation,
  name={$\mathbb{S}$},
  description={Solution set of a numerical constraint satisfaction problem or of a set inversion problem},
  sort={sa}} 
  
  
        \newglossaryentry{not:contractor}{
  type=notation,
  name={$ \mathcal{C}\big([\bm{x}]\big)$},
  description={Contractor applied to the box $[\bm{x}]$},
  sort={ca}} 
  
 
  
   

\makeglossaries
\glsaddall



